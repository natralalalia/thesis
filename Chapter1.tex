\chapter{Introduction}
\label{cha:intro}

\section{Aims of the Study}
\label{sec:aim}
What is neuromorphic vision.\\
What does neuromorphic vision recognition do.\\
What are spiking neurons.\\

\section{Why Is It Important?}
\label{sec:imp}
Exploring and mimicking invariant object recognition within the brain is a promising approach to tackling the computational difficulty;
in turn it also contributes to understanding biological visual processing by means of mimicking neural activity in the visual system of the brain.
Moreover, energy-efficiency improvements following from the great energy efficiency of biological systems will help in building object recognition systems, e.g. posture recognition for human-machine interfaces in mobile devices.  

\section{How to Model Neuromorphic Vision}
\label{sec:how}
Basic SNN description and H/W set-up.\\
Modelling the network with populations of neurons, synapses and learning rules.

\section{Contributions}
\label{sec:ctb}

\section{Publications}

\section{Thesis Structure}
\label{sec:str}