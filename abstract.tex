%%Neuromorphic Engineering
%Neuromorphic engineering uses analogue, digital, and mixed-mode VLSI and software systems to mimic the biological neural architectures. 
%It has been a hot area of research in recent years, since the start of searching alternative computer architectures other than Von Neumann.
%Human brain, as a computing system, couples the computation and the memory closely together on the neurons and the connections between them.
%Thus the biological neural architecture intrinsically solves the problem of computation-memory bottleneck.
%Moreover the brain wins over its artificial counterpart, the computer, on cognitive tasks, but using much lower computing frequency, slower communication and efficient power usage.
%Neuromorphic engineering aims to unveil the mystery of the brain approaches and also to build brain-inspired computing system.
% 
%%Spiking Neural Networks as a tool [wiki]
%Spiking neural networks (SNNs) fall into the third generation of neural network models, increasing the level of realism in a neural simulation.
%In addition to neuronal and synaptic state, SNNs also incorporate the concept of time into their operating model.
%%maybe a bit more about what is spiking neuron...
%%tool for NE
%SNNs work as a simulating tool for neuromorphic engineering, where applications can operate.
%
%%Neuromorphic Vision
%The study on computer vision has been active for several decades, while neuromorphic vision is still in its infancy.
%Although SNNs have been verified the equal capabilities of EM, and etc., there is no direct transformation from conventional methods to SNNs due to the significant different operating mechanism.
%Various coding methods exist for interpreting the outgoing spike train as a real-value number, either relying on the frequency of spikes, or the timing between spikes, to encode information. %wiki
%Comparing to computer vision, Neuromorphic vision differentiate in the input, processing of data, and learning algorithms.
%The biological-inspired vision sensors provide a main source of input data, and have been on commercial shelves after decades of research.
%Data processing and learning algorithms on SNNs are still in the initial stages.
%
%%Object Recognition application
%This thesis uses object recognition as a cut-in point to explore the Neuromorphic Vision.
%SNN models and learning algorithms have been proposed the studied in the thesis.

%TODO Abstract the work in one paragraph.
The Spiking Neural Network (SNN) has not achieved the recognition performance and learning ability of its non-spiking competitor, the Artificial Neural Network (ANN), on cognitive tasks.
Nevertheless, the intrinsic energy efficiency of SNN architecture draws continuous attraction and contribution to the field of neuromorphic engineering.
This research aims to equip SNNs with equivalent cognitive ability as ANNs' by analysing network dynamics and exploring new learning algorithms, thus to put low-cost SNN models to practical applications.

