%Abstract the work in one paragraph.
The Spiking Neural Network (SNN) has not achieved the cognitive capability and learning ability of its non-spiking competitor, the Artificial Neural Network (ANN).
Nevertheless, the intrinsic energy efficiency of SNN system draws continuous attraction and contribution to the field of neuromorphic engineering.
This research aims to equip SNNs with equivalent cognitive ability to ANNs' by analysing network dynamics and exploring new learning algorithms, thus to put SNN models to practical applications, e.g. object recognition.

%Motivation of the problem
Neuromorphic engineering has lead to the development of biologically-inspired computer architectures which may provide an alternative to the conventional Von Neumann and rival the Human brain in terms of energy efficiency and cognitive capabilities.
Although there are a number of neuromorphic platforms available for large-scale SNN simulations, programming these brain-like machines to be competent in cognitive applications still remains unsolved.
On the other hand Deep Learning arise in the research of ANNs has dominated the state-of-the-art solutions on cognitive tasks.
To catch up with ANNs' beat-human performance, this work took object recognition as the cut-in point to explore the capability and learning ability of SNNs.

%Methods
%The research started from a real-time hand posture recognition system built on a complete neuromorphic platform; it transformed the off-line training of connection weights to fit in SNNs.
%The weights transformation was then generalised to commonly used object recognition tasks in Computer Vision using a biologically plausible activation function, Noisy Softplus.
The research started from modelling neural dynamics of spiking neurons to simple but accurate activation functions using in ANNs, which enables SNN to be trained off-line just like deep learning architectures.
We then took an extra step to formalise an on-line, biological-plausible, unsupervised learning algorithm for training deep SNNs.
Moreover, a spike-based dataset was published to support fair competition between researchers and quantitatively measure progress in Neuromorphic Vision.

%TAchievements and Limitations (Strong and Weak points)
An off-line trained SNN achieved state-of-the-art classification accuracy, 98.85\% using standard leaky integrate-and-fire neuron, on MNIST dataset. 
The on-line learning algorithm performs similar and even better classification and reconstruction than ANNs'. 
The results demonstrated the equivalent performance and learning capability of spiking neurons comparing to conventional ANNs.
%In reality, the recognition and learning were time consuming due to the limitations on the rate-based encoding.
