Neuromorphic Engineering~(NE) has led to the development of biologically-inspired computer architectures which may approach the performance of the human brain in terms of energy efficiency and cognitive capabilities.
Although there are a number of neuromorphic platforms available for large-scale Spiking Neural Network~(SNN) simulations, the problem of programming these brain-like machines to be competent in cognitive applications still remains unsolved.
On the other hand, Deep Learning has emerged in Artificial Neural Network (ANN) research to dominate state-of-the-art solutions for cognitive tasks.
Thus the main research problem emerges of understanding how to operate and train biologically-plausible SNNs to close the gap in cognitive capabilities between SNNs and ANNs.

SNNs can be trained by first training an equivalent ANN and then transferring the tuned weights to the SNN.
This method is called `off-line' training, since it does not take place on an SNN directly, but rather on an ANN instead.
However, previous work on such off-line training methods has struggled in terms of poor modelling accuracy of the spiking neurons and high computational complexity.
In this thesis we propose a simple and novel activation function, Noisy Softplus~(NSP), to closely model the response firing activity of biologically-plausible spiking neurons,
and introduce a generalised off-line training method using the Parametric Activation Function~(PAF) to map the abstract numerical values of the ANN to concrete physical units, such as current and firing rate in the SNN.
Based on this generalised training method and its fine tuning, we achieve the state-of-the-art accuracy on the MNIST classification task using spiking neurons, 99.07\%, on a deep spiking convolutional neural network~(ConvNet).


We then take a step forward to `on-line' training methods, where Deep Learning modules are trained purely on SNNs in an event-driven manner.
Existing work has failed to provide SNNs with recognition accuracy equivalent to ANNs due to the lack of mathematical analysis. 
Thus we propose a formalised Spike-based Rate Multiplication~(SRM) method which transforms the product of firing rates to the number of coincident spikes of a pair of rate-coded spike trains.
%possibility of some specific firing events of a pair of rate-coded spike trains.
Moreover, these coincident spikes can be captured by the Spike-Time-Dependent Plasticity~(STDP) rule to update the weights between the neurons in an on-line, event-based, and biologically-plausible manner.
Furthermore, we put forward solutions to reduce correlations between spike trains; thereby addressing the result of performance drop in on-line SNN training. 
The promising results of spiking Autoencoders~(AEs) and Restricted Boltzmann Machines~(SRBMs) exhibit equivalent, sometimes even superior, classification and reconstruction capabilities compared to their non-spiking counterparts.

To provide meaningful comparisons between these proposed SNN models and other existing methods within this rapidly advancing field of NE, we propose a large dataset of spike-based visual stimuli and a corresponding evaluation methodology to estimate the overall performance of SNN models and their hardware implementations.
