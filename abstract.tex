%Abstract the work in one paragraph.
The Spiking Neural Network (SNN) has not achieved the cognitive capability and learning ability of its non-spiking counterpart, the Artificial Neural Network (ANN).
Nevertheless, the intrinsic energy efficiency of the SNN system continues to draw attention towards the field of neuromorphic engineering.
The research described in this thesis aims to equip SNNs with equivalent cognitive ability to ANNs by analysing network dynamics and exploring new learning algorithms, thus putting SNN models into practical applications, such as object recognition.

%Motivation of the problem
Neuromorphic engineering has led to the development of biologically-inspired computer architectures which may provide an alternative to the conventional Von Neumann architecture and achieve the performance of human brain in terms of energy efficiency and cognitive capabilities.
Although there are a number of neuromorphic platforms available for large-scale SNN simulations, programming these brain-like machines to be competent in cognitive applications still remains unsolved.
On the other hand Deep Learning has emerged in ANN research to dominate state-of-the-art solutions for cognitive tasks.
Thus it raises the main research problem of how to operate and train biologically-plausible SNNs to make them as competent as ANNs.
%To catch up with ANNs' better-than-human performance, this work takes object recognition as the cut-in point to explore the capability and learning ability of SNNs.

%Methods
%The research started from a real-time hand posture recognition system built on a complete neuromorphic platform; it transformed the off-line training of connection weights to fit in SNNs.
%The weights transformation was then generalised to commonly used object recognition tasks in Computer Vision using a biologically plausible activation function, Noisy Softplus.
The research starts from modelling the neural dynamics of spiking neurons to create simple but accurate activation functions for use in ANNs, which enables the SNN to be trained off-line just like Deep Learning architectures.
We then take an extra step to formalise an on-line, biologically-plausible, unsupervised learning algorithm for training deep SNNs.
In addition, a spike-based dataset was created and published to support fair competition between researchers and quantitatively measure progress in Neuromorphic Vision.

%TAchievements and Limitations (Strong and Weak points)
An off-line trained SNN achieves state-of-the-art classification accuracy (99.07\%) on the MNIST dataset, using standard leaky integrate-and-fire neurons.
An on-line learning algorithm performs similar, and sometimes better, classification and reconstruction than an ANN. 
The results demonstrate the equivalent performance and learning capability of spiking neurons compared to conventional ANNs.
%In reality, the recognition and learning were time consuming due to the limitations on the rate-based encoding.
