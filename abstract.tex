%Abstract the work in one paragraph.
The Spiking Neural Network (SNN) has not achieved the cognitive capability and learning ability of its non-spiking competitor, the Artificial Neural Network (ANN).
Nevertheless, the intrinsic energy efficiency of SNN system draws continuous attraction and contribution to the field of neuromorphic engineering.
This research aims to equip SNNs with equivalent cognitive ability as ANNs' by analysing network dynamics and exploring new learning algorithms, thus to put SNN models to practical applications, e.g. object recognition.

%Motivation of the problem
Neuromorphic engineering has lead towards the development of biologically-inspired computer architectures which may alternate conventional Von Neumann and rival the Human brain in terms of energy efficiency and cognitive capabilities.
Although there are a number of neuromorphic platforms available for large-scale SNN simulations, how to operate these brain-like machines to be competent in cognitive applications still remains unsolved.
While Deep Learning within the research of ANNs has dominated the state-of-the-art solutions on cognitive tasks.
To catch up with ANNs' beat-human performance, this work took object recognition as the cut-in point to explore the capability and learning ability of SNNs.

%Methods
This research started from a real-time hand posture recognition system built on a complete neuromorphic platform.
This work transformed the off-line trained connection weights to fit in SNNs.
The weights transformation was then generalised to common used object recognition tasks in Computer Vision by using a biologically plausible activation function, Noisy Softplus.
This work extended to a formalised on-line learning algorithm for spiking deep neural networks.
Moreover, a spike-based dataset was published to support fair competition between researchers and quantitatively measure progress in Neuromorphic Vision.

%TAchievements and Limitations (Strong and Weak points)
The results demonstrated the close recognition performance of SNNs comparing to conventional ANNs thanks to the effective weights transformation, and the equivalent learning capability of spiking neurons to build up deep architectures.
%In reality, the recognition and learning were time consuming due to the limitations on the rate-based encoding.
